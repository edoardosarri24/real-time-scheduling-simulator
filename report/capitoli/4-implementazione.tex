\chapter{Implementazione}
\label{cap:implementazione}
L'implementazione deve rappresentare una simulazione deterministica. Per questo motivo non sono stati usati thread (oggetti Java \texttt{Thread}), la gestione del tempo avviene in modo discreto.

%%%%%%%%%%%%%%%%%%%%%%%%%%%%%%%%%%%%%%%%%%%%%%%%%%%%%%%%%%%%
\section{Scheduler}
Ogni possibile implementazione di un algoritmo di scheduling è un'estensione della classe \texttt{Scheduler}. Questo permette di definire i comportamenti comuni a tutti gli scheduler e di astrarre future implementazioni.

Per l'implementazione, oltre a ciò che definisce uno scheduler (i.e., taskSet e l'eventuale protocollo di accesso alle risorse), è stato necessario mantenere una lista di task pronti \texttt{readyTasks} e una di task bloccati \texttt{blockedTask}. Inoltre è necessario mantenere un riferimento all'ultimo task che è andato in esecuzione.

\myskip

L'implementazione della logica di scheduling è la stessa sia per RM che per EDF: il metodo \texttt{schedule} con i metodi helper necessari sono stati, per questo motivo, inseriti all'interno della classe base. Ogni implementazione concreta deve solo definire alcuni aspetti usati poi nella logica. Per modellare questo principio è stato usato il Template Method pattern: il metodo \texttt{schedule} è dichiarato pubblico e \texttt{final} in modo che non possa essere modificato dalle implementazioni, mentre gli hooks chiamati al suo interno sono dichiarati \texttt{abstract} e \texttt{protected}.

\myskip

Quando uno scheduler viene creato oltre a assegnare il taskset e il protocollo di accesso alle risorse, vengono inizializzate le strutture relative al protocollo. La scelta di fare questo assegnamento qua e non nel costruttore della classe dedicata al protocollo è dovuta alle dipendenze: per come è stato implementato il sistema, l'oggetto principale (e anche l'ultimo che deve essere istanziato) è lo scheduler, e quindi il protocollo si basa su di esso.

Per gestire i tempi rilevanti, cioè quelli in cui lo scheduler deve prendere il controllo del sistema per rivedere la propria politica di scheduling, è delegata alla struttura \texttt{readyTasks}. Nell'intervallo tra un periodo e il successivo infatti lo scheduler non fa altro che mandare in esecuzione uno dopo l'altro il task a priorità maggiore. Gli eventi rilevanti sono l'unione ordinata dei multipli di ciascun periodo fino al minimo comune multiplo dei periodi oppure fino a 10 volte il periodo maggiore (per semplicità del caso sia molto oneroso generare questa lista).

Per capire i passaggi che esegue la simulazione consideriamo e analizziamo il sequence diagram di Figura~\ref{fig:sequenceDiagram}:
\begin{itemize}
    \item \texttt{assignPriority} \\
        Assegna le priorità secondo l'implementazione in questione. È dichiarato astratto in \texttt{Scheduler} e deve essere implementato dalle classi concrete.
    \item \texttt{MyClock.reset} \\
        Resettando il clock di sistema prima di iniziare la simulazione, si permette di eseguire più volte il metodo \texttt{schedule} all'interno dello stesso main. Questo è necessario perché, come viene descritto nella Sezione~\ref{subsec:clock}, il clock del sistema è implementato in maniera statica.
    \item \texttt{initStructure} \\
        Inizializza le strutture dati usate dalla simulazione: la lista di task pronti, e la lista di eventi importanti.
    \item \texttt{checkFeasibility} \\
        Controlla se è possibile, o meglio se non è possibile (e.g., per RM), schedulare il taskset secondo i test di schedulabilità implementati: per RM utilizza l'hyperbolic bound; per EDF valuta se il fattore di utilizzo è maggiore o minore dell'unità.
    \item \texttt{Task.execute} \\
        È il metodo che manda in esecuzione il sistema per un tempo limitato. Questo tempo va dal tempo attuale fino al prossimo evento significativo.
    \item \texttt{access-progress-release} \\
        Sono i metodi che definiscono il protocollo di accesso alle risorse. I dettagli implementativi sono specificati nel Paragrafo~\ref{sec:resaccprot}.
    \item \texttt{Chunk.execute} \\
        Definisce l'esecuzione del singolo chunk. In particolare si occupa della fase di logging.
    \item \texttt{releasePeriodTasks} \\
        Si occupa di rilasciare i task nel momento in cui scocca il suo periodo. Inoltre controlla che quei task abbiano finito di eseguire ed eventualmente solleva una \texttt{DeadlineMissException}.
    \item \texttt{reset} \\
        Durante l'esecuzione lo stato dei task e dei chunk cambia per riflettere informazioni relative all'esecuzione. Con questo metodo si vuole riportare i task e chunk al loro stato iniziale.
\end{itemize}

\begin{figure}[htbp]
    \centering
    \includegraphics[width=.65\textwidth]{immagini/sequence diagram.pdf}
    \caption{Sequence Diagram scheduling}
    \label{fig:sequenceDiagram}
\end{figure}

\subsection{Rate Monotonic}
Vediamo come sono implementati i metodi astratti della classe \texttt{Scheduler} in RM, cioè quelli che definiscono il suo comportamento rispetto agli altri scheduler.
\begin{itemize}
    \item \texttt{checkFeasibility} \\
        Viene controllato l'hyperbolic bound. È stato deciso di non considerare il test di Liu \& Layland visto che questo non è tight: se un taskset non risulta schedulabile secondo questo test, l'hyperbolic bound potrebbe definire che è schedulabile.
    \item \texttt{assignPriority} \\
        Assegna la priorità in modo inverso rispetto alla durata del periodo.
    \item \texttt{addReadyTask} \\
        Siccome la lista dei task pronti per l'esecuzione è ordinata in modo dinamico (visto che è implementata come un $TreeSet<Task>$ con un comparatore) secondo la priorità dinamica, aggiunge semplicemente il task a tale lista.
\end{itemize}

\subsection{Earliest Deadline First}
Vediamo come sono implementati i metodi astratti della classe \texttt{Scheduler} in EDF, cioè quelli che definiscono il suo comportamento rispetto agli altri scheduler.
\begin{itemize}
    \item \texttt{checkFeasibility} \\
        Viene controllato che il fattore di utilizzo del taskset sia minore o uguale a 1.
    \item \texttt{assignPriority} \\
        Assegna la priorità in modo inverso rispetto alla durata della deadline.
    \item \texttt{addReadyTask} \\
        Aggiunge il task alla lista dei task pronti e poi la riordina secondo la prossima deadline. Il metodo che stabilisce la prossima deadline è \texttt{Task.nextDeadline}.
\end{itemize}

\subsection{Test}
Per quanto riguarda i test su RM sono stati eseguiti con e senza risorse condivise. In entrambi i casi oltre a garantire che non si arrivi a un deadline miss quando il taskset è schedulabile e ci si arrivi invece quando il taskset non lo è, è stato osservato il file di log generato per capire se esso era compliant alla politica di scheduling con e senza PCP.

Per eseguire i test su taskset non schedulabili è stato preso come SUT (System Under Test) con fattore di utilizzo maggiore di 1. In questo modo siamo sicuri che il taskset non sia schedulabile.

\myskip

Per EDF valgono gli stessi concetti di cui sopra. Ovviamente siccome non è stato implementato un protocollo di accesso alle risorse da usare con EDF, sono stati testati solo taskset i cui chunk non utilizzano risorse condivise.

%%%%%%%%%%%%%%%%%%%%%%%%%%%%%%%%%%%%%%%%%%%%%%%%%%%%%%%%%%%%
\section{Resource Access Protocol}
\label{sec:resaccprot}
Ogni implementazione di un protocollo di accesso alle risorse deve estendere la classe \texttt{ResourceProtocol}.

L'idea iniziale era di implementare il concetto astratto di protocollo di accesso tramite un'interfaccia, ma vista la necessità di mantenere un riferimento allo scheduler nel protocollo si è introdotto questo campo nella classe base.

\myskip

I metodi definiti da questa classe astratta sono le operazioni che devono essere svolte da un protocollo di questo tipo: deve gestire la fase di accesso, progresso e rilascio. Definisce anche il metodo \texttt{initStructures} che ha il compito di inizializzare le strutture dati usate dal protocollo.

Oltre a questi metodi è dichiarato un metodo \texttt{initStructures} che inizializza le strutture necessarie al protocollo.

\subsection{Priority Ceiling Protocol}
Tralasciando quello che fanno i metodi di accesso, progresso e rilascio, che riflettono quanto ci dice la teoria, in questa classe le strutture usate sono prevalentemente due:
\begin{itemize}
    \item \texttt{ceiling} \\
        È una mappa che associa ad ogni risorsa il suo ceiling, cioè la massima priorità nominale dei task che usano quella risorsa.
    \item \texttt{busyResources}\\
        È una lista delle risorse che sono occupate da un qualche task.
\end{itemize}

%%%%%%%%%%%%%%%%%%%%%%%%%%%%%%%%%%%%%%%%%%%%%%%%%%%%%%%%%%%%
\section{Fault injection}

\subsection{Additional execution time}
Per implementare il fault injection di un additional execution time in un \texttt{chunk} è stato previsto un altro costruttore che prendesse, oltre ai parametri previsti, anche un \texttt{overheadExecutionTime}.

Il motivo per non introdurre una classe che eseguisse l'injection è stato che un chunk nella realtà nasce con il suo execution time e non ci sono altre entità che aumentano quello campionato; il costruttore vuole modellare questa idea.

\subsection{Priority Ceiling Protocol}
Sono state aggiunte due classi che introducono due tipi di fault: il primo setta una priorità dinamica errata quando un task entra nella critical section; il secondo non fa acquisire una risorsa quando al chunk che la utilizza.

\myskip

Le classi che implementano questi comportamenti sono:
\begin{itemize}
    \item \texttt{PriorityCeilingProtocolFaultSetPriority} \\
        L'idea è campionare un valore delta, da aggiungere poi alla priorità dinamica corretta, da una distribuzione uniforme. Tale distribuzione è un'istanza di \texttt{UniformSampler} della libreria Sirio. I due parametri necessari alla distribuzione sono presi tramite costruttore.
    \item \texttt{PriorityCeilingProtocolFaultAcquireResource} \\
        Si campiona da una distribuzione uniforme (i.e., \texttt{Math.random()}) un valore compreso tra 0 e 1. Se il valore passato come soglia è minore di questo valore campionato allora si salta il blocco di codice responsabile dall'acquisizione del semaforo. Per non rilasciarlo al termine dell'esecuzione del chunk, questo viene inserito in una lista.
\end{itemize}

%%%%%%%%%%%%%%%%%%%%%%%%%%%%%%%%%%%%%%%%%%%%%%%%%%%%%%%%%%%%
\section{Utility}
Vediamo la scelta su alcuni componenti di utilità.

\subsection{Logging}
Per il logging è stato implementato un semplice logging su un file e viene rappresentato come una sequenza di coppie $<evento,tempo>$.

Il file di destinazione delle tracce loggate è \texttt{trace.log}.

\subsection{Clock}
\label{subsec:clock}
Il tempo all'interno del sistema è gestito staticamente, e quindi a livello globale. Questa scelta è dovuta al fatto che praticamente tutti gli oggetti devono accedere al clock del sistema; in questo modo si evita di passarlo ogni volta nei vari metodi chiamati a cascata. L'implementazione è stata fatta tramite il pattern Singleton.

Il clock del sistema è rappresentato dalla classe \texttt{MyClock}. Questa non fa altro che mantenere il tempo assoluto ed esporre due metodi che permettono di avanzare di un dato intervallo temporale e avanzare fino a un determinato tempo.

\myskip

Il tempo è gestito tramite oggetti di tipo \texttt{Duration}, classe di \texttt{java.time} che implementa oggetti immutabili e che permette una facile gestione del tempo.

In particolare il tempo deve essere considerato dall'utente (i.e., passato in input e restituito poi in output) in millisecondi, ma il sistema lo gestisce tramite i nanosecondi. Questo permette di lavorare con millisecondi frazionari quando si campiona dalle distribuzioni di Sirio.

La stampa all'interno del file di log \texttt{trace.log} nel formato corretto è implementata nel metodo \texttt{Utils.printCurrentTime}.

\subsection{Sampling dei tempi}
Quando si deve definire i tempi che definiscono i vari componenti del sistema, cioè come il periodo, la deadline, l'execution time di un chunk, si usa un campionamento da una data distribuzione.

Le distribuzioni sono implementate dalla libreria \texttt{Sirio}; oltre a quelle definite dalla libreria è stata implementata la classe \texttt{ConstantSampler}, che permette di gestire tempi costanti, mantenendo l'astrazione della libreria Sirio.

\myskip

I sampler di Sirio restituiscono oggetti di tipo \texttt{BigDecimal}. Come detto nel paragrafo sopra il sistema gestisce il tempo come oggetti di tipo \texttt{Duration}. Per implementare questo meccanismo è stata implementata la classe \texttt{SampleDuration}, che preso un \texttt{Sampler} restituisce il rispettivo tempo in nanosecondi.

\subsection{Eccezioni}
Durante l'esecuzione si possono verificare dei problemi, più o meno previsti. Questi sono gestiti tramite eccezioni; questo permette in futuro di cambiare o aggiungere un comportamento del sistema quando si verificano determinate situazioni.

\myskip

Le eccezioni implementate e utili sono:
\begin{itemize}
    \item \texttt{DeadlineMissedException} \\
        Viene sollevata quando un task non rispetta la deadline. Il sistema non la gestisce, ma la propaga fino al main: in questo modo se e quando si verifica questo problema, viene stampata in \texttt{trace.log} e la simulazione si arresta.
    \item \texttt{AccessResourceProtocolException} \\
        Viene sollevata quando un task viene bloccato dal metodo \texttt{access} del protocollo di accesso.
\end{itemize}