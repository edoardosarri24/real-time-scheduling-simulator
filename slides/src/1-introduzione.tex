\section{Introduzione}
\subsection{\href{https://github.com/edoardosarri24/real-time-scheduling-simulator.git}{GitHub}}

\begin{frame}{Introduzione}
    \begin{block}{Obiettivo}
        Simulare l'esecuzione di un taskset secondo un dato algortimo di scheduling e protocollo di accesso alle risorse.
    \end{block}
\end{frame}

\begin{frame}{Introduzione}
    \begin{block}{Output}
        Un file di log contenente una traccia di esecuzione, cioè una sequenza di coppie $<tempo,evento>$, dove i possibili eventi sono:
        \begin{itemize}
            \item Rilascio di un job di un task.
            \item Acuisizione e rilascio di una risorsa da parte di un chunk.
            \item Completamento dell'esecuzione di un chunk o di un job di un task.
            \item Preemption su un task.
        \end{itemize}
    \end{block}
\end{frame}

\begin{frame}{Introduzione}
    \begin{block}{Capacità}
        \begin{itemize}
            \item Utilizzare Rate Monotonic con e senza risorse convidese insieme a Priority Ceiling Protocol.
            \item Utilizzare Earliest Deadline First senza risorse condivise.
            \item Rilevare eventuali deadline miss.
            \item Introdurre in modo stocastico e rilevare un additional execution time in un chunk.
            \item Introdurre un fault a livello del protocollo di accesso alle risorse per cui PCP imposta male la priorità dinamica dei task.
            \item Introdurre un fault a livello di chunk per cui esso non acquisisce (e rilascia) il semaforo della risorsa che userà.
        \end{itemize}
    \end{block}
\end{frame}