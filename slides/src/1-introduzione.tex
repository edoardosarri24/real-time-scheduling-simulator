\section{Introduzione}
\subsection{\href{https://github.com/edoardosarri24/real-time-scheduling-simulator.git}{GitHub}}

\begin{frame}{Introduzione}
    \begin{block}{Obiettivo}
        Simulare l'esecuzione di un taskset secondo un dato algortimo di scheduling e un protocollo di accesso alle risorse.
    \end{block}
\end{frame}

\begin{frame}{Introduzione}
    \begin{block}{Output}
        Un file di log contenente la traccia di esecuzione, cioè una sequenza di coppie $<tempo,evento>$, dove i possibili eventi sono:
        \begin{itemize}
            \item Rilascio di un task.
            \item Acuisizione e rilascio di una risorsa da parte di un chunk.
            \item Completamento dell'esecuzione di un chunk o di un job di un task.
            \item Preemption su un task.
            \item Deadline miss di un job di un task.
            \item Fault di un chunk.
        \end{itemize}
    \end{block}
\end{frame}

\begin{frame}{Introduzione}
    \begin{block}{Capacità}
        \begin{itemize}
            \item Generare una traccia di esecuzione.
            \item Genenare un dataset di tracce di esecuzione.
            \item Simulare Rate Monotonic con e senza risorse convidese insieme a Priority Ceiling Protocol.
            \item Simulare Earliest Deadline First senza risorse condivise.
            \item Rilevare eventuali deadline miss.
            \item Controllare la feasibility di un taskset dato il relativo algortimo di scheduling.
        \end{itemize}
    \end{block}
\end{frame}

\begin{frame}{Introduzione}
    \begin{block}{Capacità (fault injection)}
        \begin{itemize}
            \item Introdurre in modo stocastico e rilevare un additional execution time in un chunk.
            \item Introdurre un fault a livello del protocollo di accesso alle risorse per cui PCP imposta male la priorità dinamica dei task.
            \item Introdurre un fault a livello di chunk per cui esso non acquisisce (e rilascia) il semaforo della risorsa che userà.
        \end{itemize}
    \end{block}
\end{frame}

\begin{frame}{Introduzione}
    \begin{block}{Utilizzo}
        \begin{itemize}
            \item All'interno del main devono essere definiti i componenti necessari: \texttt{Resource}, \texttt{Chunk}, \texttt{Task}, \texttt{TaskSet}, \texttt{Scheduler} e \texttt{ResourceProtocol}.
            \item Per avviare una simulazione chiamare il metodo \texttt{schedule} o \texttt{scheduleDataset} di uno \texttt{Scheduler}.
            \item I tempi devono essere passati e letti dal sistema in millisecondi. Il sistema li elabora in nanosecondi per una maggiore precisione.
        \end{itemize}
    \end{block}
\end{frame}