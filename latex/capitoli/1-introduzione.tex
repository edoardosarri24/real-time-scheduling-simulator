\chapter{Introduzione}
Il progetto vuole modellare e implementare un sistema in Java che permetta di generare tracce di un'esecuzione a partire dalla definizione di un taskset con o senza risorse da usare in mutua esclusione. Le eventuali risorse sono gestite da un protocollo di accesso alle risorse.

Ogni traccia è definita come una sequenza di coppie $<tempo, evento>$, dove un $evento$ può essere: rilascio di un job di un task; acquisizione/rilascio di un semaforo da parte di un job di un task; completamento di un chunk; completamento di un job di un task.

%%%%%%%%%%%%%%%%%%%%%%%%%%%%%%%%%%%%%%%%%%%%%%%%%%%%%%%%%%%%%%%%
\section{Capacità}
A partire da un taskset, il sistema ha le capacità di:
\begin{itemize}
    \item Generare la traccia di esecuzione del taskset schedulato tramite un dato algoritmo di scheduling e ed un eventuale protocollo di accesso alle risorse.
    \item Rilevare eventuali deadline miss. Se al termine del prorpio periodo, un task non ha completato un tutti i chunk di cui è composto allora il sistema lo evidenzia e si arresta.
    \item Introdurre e rilevare un additional execution time in un chunk. Questo modella un tempo di computazione di un chunk maggiore di quello che ci aspettiamo, cioè maggiore del WCET.
\end{itemize}